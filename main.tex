\documentclass{article}
\usepackage{graphicx} % Required for inserting images
\usepackage{lipsum}
\usepackage[a4paper]{geometry}
\usepackage{amsmath}
\newcommand\dollars{\$}

\newenvironment{itquote}
  {\begin{quote}\itshape}
  {\end{quote}\ignorespacesafterend}
\newenvironment{itpars}
  {\par\itshape}
  {\par}

\title{Learning LaTeX}
\author{Muhammad Sahal Mulki}
\date{August 2024}

\begin{document}

\maketitle

\section{Introduction}

I'm Sahal and I'm trying to learn \LaTeX! Most of the content in this document is gonna be about me, my projects or just placeholder jibberish. Now let me tell you something I found really interesting. $a^2 + b^2 = c^2$ This is Pythagoras' theorem, and it suggests that given a triangle, and it's sides $a$, $b$, and $c$, where $c$ is the hypotenuse, the value of $a^2 + b^2$ is going to be the same as $c^2$. Furthermore, to extract the true value of $c$, one may use the expression $\sqrt{c}$. Truly fascinating.

\bigskip

Now let us get the tenth of $c$, which can be written as

$$\frac{\sqrt{a^2 + b^2}}{10}$$

\bigskip

AMAZINGLY FASCINATING\dots

\bigskip

I like counting things, like what kind of fruits I have in my kitchen. Let's start.

\begin{enumerate}
    \item An apple.
    
    \item A banana.

    \item Mayonnaise.

    \item $\infty$ Oranges.
    
\end{enumerate}

\lipsum[1]

\section{Calculating how much money is gonna be spent on OpenAI's API}

I have a project where I need to use the OpenAI API. Let us calculate how much money that's going to take me. This is my input prompt for the gpt-4o API:

\bigskip

\begin{itpars}You are an Image OSINT Investigator. I will provide you with an image and it's your job to determine where it was taken. By the end of your investigation you have to have at least a rough guess of the image's location. DECIDE ON ONE FINAL LOCATION, EVEN IF IT'S WRONG, JUST GUESS.

\smallskip

Guess where this image was taken, get creative.Here's some AI-generated addresses that could be wrong to get you started. Choose the best out of these if you're uncertain about the image: 
\begin{enumerate}

\item Acre Subdistrict, North District, Israel
\item Acre Subdistrict, North District, Israel 
\item Acre Subdistrict, North District, Israel
\item Acre Subdistrict, North District, Israel
\item Acre Subdistrict, North District, Israel
\end{enumerate}
\end{itpars}

That is $163$ tokens, which is roughly \dollars0.000815. Following this will be an image with the size $500$ by $500$ which is \dollars0.001275. Finally, let's assume an output of $150$ tokens which is \dollars0.00225. Finally let us calculate the total amount of money a single call would take:

\bigskip

$$ 0.000815 + 0.001275 + 0.00225 = 0.00434$$

\bigskip


\dollars0.00434 is equal to 0.016 United Arab Emirate Dirhams (hereafter referred to as AED). Now let us estimate the amount of money 2000 API calls would take, which is equal to:

\bigskip


$$0.016 \cdot 2000 = AED32$$

\bigskip


32 Dirhams it would take if I was to do a normal API call. But I decide to use the OpenAI Batch API, which has a 50\% discount on every API call, with the disadvantage of taking 24 hours to process a single batch.

\bigskip

$$ AED32 \cdot .5 = AED26$$
\bigskip

Ok now check this sick matrix out.

$$\begin{bmatrix}
1 & 2 & 3\\
4 & 5 & 6\\
\end{bmatrix}
$$

Really sick.

\section{Images.}

I'm new to \LaTeX and don't know how to use images in a document. Let me try now.

\smallskip
\begin{center}
\includegraphics[scale = .4]{building.jpg}
\end{center}

\section{Text formatting}

Text like \textbf{this} can be formatted in \textit{many} \underline{ways}. Now time for a \textbf{\textit{\underline{table.}}}

\newpage

\begin{table}[]
\caption{A nice table.}
\begin{center}
    \begin{tabular}{|c|c|}
    \hline1 & 2 \\
    \hline3 & 4 \\
    \hline
    \end{tabular}
\end{center}
\end{table}

\end{document}
